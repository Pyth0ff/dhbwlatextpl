\chapter{{\LaTeX}-Vorlage}
Das  hier ist eine kleine {\LaTeX}-Vorlage fuer die DHBW Loerach.
Speziell fuer den fachbereich Informatik.

\section{Danke Dounald}
Dank Dounald Knuth\cite{knuth} der Erfinder von \TeX~ koennen wir hier ganz schoen, tolle und Fancy Ausarbeitungen schreiben ohne einen gedanken an das Formatieren zu verschweden.


\newpage
\section{Aufbau der Vorlage}
Die Vorlage ist von der Standart Vorlage book.cls abgeleitet und ist direkt auf DIN A4 gesetzt.
Auch die Schriftart ist auf 12pt und 1,5 Zeilebabstand gesetzt.

\section{Beispiele}
Nun sollen einzelne Beispiele folgen, wie Bilder, Tabelle, Auflistungen, Programmcode, Zitate 

\subsection{Bilder}
ein Bild wird ganz einfach eingebunden, darum herum wird ein \textit{figure}  gesetzt damit die Abbildung auf auf der Abbildungsauflistung erscheint.\\
Mit \textit{caption} wird eine Abbildungsunterschrift angegeben und mit \textit{label} wird ein label vergeb mit dem man auf eine Abbildung verweisen kann.
\begin{figure}[hb] \centering
  \includegraphics[width=3cm]{./img/tux.png}
  \caption{eine Bild Unterschrift}
  \label{tux_1}
\end{figure}
Als Beispiel eines label kann in \ref{tux_1} gesehen werden das dort ein Pinguin abgebildet wird.


\subsection{Tabelle}
Eine Tabelle wird wie bei einer Abbildung wie folgt dargestellt: \\

\begin{table}[hb] \centering
\begin{tabular}{|r|r|}
\hline
Zahl & Quatsch \\
\hline
1  &  0  \\
\hline
2  &  1  \\
\hline
3  &  2  \\
\hline
4  &  3  \\
\hline

\end{tabular}

\caption{eine QuatschTabelle}
\label{tabquatsch}
\end{table}



\section{Glossar}
Das Glossar listet alle Verwendeten Begriffe auf und verlinkt diese auch im Text. Da ist \gls{LaTeX} echt cool und es macht spass mit einem \gls{computer} zuarbeiten.



