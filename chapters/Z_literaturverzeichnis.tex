\newpage

\markboth{}{}%
\renewcommand{\baselinestretch}{.90}

%Quellen
\begin{thebibliography}{20}
%Reihenfolge: Personen, Internes  B"ucher, Magazine, Internetquellen
%%%%%%%%%   Personen 
\bibitem{myComp}
	Interne Dokumente, Marktanalysen, Strategiepläne und Gespräche mit Mitarbeitern\\
	aus dem eigenen Unternehmen
	<MONAT><jahr>, <Unternehmen Name, ORT>

%%%%%%%%%   Vorlesungsunterlagen
\bibitem{vl}
	Vorlesungsunterlagen <VL-Titel> \\
	Dozent: <Titel Name> \\
	<von> - <bis>,  DHBW-Lörrach

%%%%%%%%%   Buecher 
\bibitem{book}
	<autor> \\
	<Buch Titel> (1962)\\
	<Verlagsname>; Auflage: <wievielete Auflage><ErscheinungsDatum> \\
	ISBN-10: <ISBN Nummer>
	
\bibitem{knuth}
	Donald E. Knuth\\
	The Art of Computer Programming, Volumes 1-4\\
	Addison-Wesley Longman, Amsterdam; Auflage: 3rd edition. (3. März 2011)\\
	ISBN-10: 0321751043




%%%%%%%%%    Magazine
\bibitem{ix_110}
	ix - Magazin fuer Profesionelle Informationstechnik  \\
	1. Ausgabe Januar 2010, Seite 42\\
	Heise Zeitschriften Verlag GmbH \& Co. KG 

%%%%%%%%%    Internetquellen Paper
\bibitem{blogfefe}
	Blog von Felix von Leiter(fefe)\\
	Debian stolpert mal wieder über einen schlechten Zufallszahlengenerator.  \\
	Einsichtnahme: 30.04.2012  \\
	\url{http://blog.fefe.de/?ts=b1628b67}
   
 

\end{thebibliography}
