% everything is in seperate documentClass
\documentclass{./tpl/styles/dhbook}                                         	

%%%%%%%%%%%%%%%%%%%%%%% Definitionen bzgl. der Arbeit %%%%%%%%%%%%%%%%%%%%%%%
\dhlogo{./img/dhbw_logo.png}
\type{1. Projektarbeit}           
\title{MeinThema}
\subtitle{Themenuntertitel}
\author{Max Mustermann}
\kurs{Muster11}
\studiengang{Informationstechnik}
\fspo{Bachelor of Engenering, Praxismodul I}
\date{\today}
\schoolSupervisor{Prof. Schnattinger}
\companyName{Muster GmbH}
\companyAddress{Musterhausen}
\companyContact{MyCompany\\
		MeineStrasse 23\\
		87654 NeuHausen\\
		Tel: 05421/4324-22\\
		Web: \url{mycompany.de}}
\companyLogo{./img/mycompany_logo.png}
\companySupervisor{Prof. Dr. Donald Ervin Knuth}

\issperrvermerk


%%%%%%%%%%%%% title, Index, Abkürzungsverzeichnis und Glossar erstellen %%%%%%%%%%%%



\makeindex
\makeglossaries

%\newglossaryentry{<label>}{<settings>}

\newglossaryentry{computer}
{
  name=computer,
  description={is a programmable machine that receives input,
               stores and manipulates data, and provides
               output in a useful format}
}

\newglossaryentry{LaTeX}
{
  name=\LaTeX,
  description={is a document markup language and document preparation system for the TeX typesetting program. Within the typesetting system, its name is styled as LaTeX. The term LaTeX refers only to the language in which documents are written, not to the editor used to write those documents. }
}


%%%%%%%%%%%%%%%%%%%%%%%%%%%%%%% PDF-Optionen %%%%%%%%%%%%%%%%%%%%%%%%%%%%%%%%
\hypersetup{
	bookmarksopen=false,
	bookmarksnumbered=true,
	bookmarksopenlevel=0,
	pdftitle=\valueTitle,
	pdfsubject=\valueSubtitle,
	pdfauthor=\valueAuthor,
	pdfborder=0,
	pdfcreator=\valueAuthor,
	pdfstartview=Fit,
	pdfpagelayout=SinglePage
}

\begin{document}


	% Aktiviere blankes Seitenlayout
    \pagestyle{fancy}	   
    % Roemische Seitenzahlen
    \pagenumbering{Roman}
    
    % erzeuge Titelseite + ggf Sperrvermerk
    \maketitle
    
	
    % Zusammenfassungen
    

\phantomsection
\addcontentsline{toc}{chapter}{Zusammenfassung}%
\chapter*{Zusammenfassung}%
\label{sec:zusammenfassung-ger}%
\pdfbookmark[1]{Deutsch}{sec:zusammenfassung-ger}%
%\addcontentsline{toc}{subsection}{Deutsch}

 Lorem ipsum dolor sit amet, consectetur adipiscing elit. Morbi sed diam mauris. Aenean eleifend ultricies nibh sit amet pharetra. Aenean non tellus libero, commodo sodales quam. Curabitur id nunc quis massa euismod lacinia. Nunc eu nibh augue, at suscipit nisl. Cras imperdiet, massa ornare volutpat dapibus, nisi risus adipiscing quam, vel pharetra metus leo id risus. Fusce sit amet quam odio. Vivamus ultrices egestas erat, sit amet pellentesque nisl feugiat a. Morbi tempor lorem ac tellus auctor euismod.  

\par

\newpage  

\chapter*{Summary}%
\label{sec:zusammenfassung-eng}%
\pdfbookmark[1]{Englisch}{sec:zusammenfassung-eng}%
 Lorem ipsum dolor sit amet, consectetur adipiscing elit. Morbi sed diam mauris. Aenean eleifend ultricies nibh sit amet pharetra. Aenean non tellus libero, commodo sodales quam. Curabitur id nunc quis massa euismod lacinia. Nunc eu nibh augue, at suscipit nisl. Cras imperdiet, massa ornare volutpat dapibus, nisi risus adipiscing quam, vel pharetra metus leo id risus. Fusce sit amet quam odio. Vivamus ultrices egestas erat, sit amet pellentesque nisl feugiat a. Morbi tempor lorem ac tellus auctor euismod.  
\par 


    % einfuegen des Vorwortes
    
\phantomsection
\chapter*{Vorwort}%
\addcontentsline{toc}{chapter}{Vorwort}%
Ganz vielen Dank an alle! Und vor allem an die, die mich unterstützt haben.

    
    % erzeuge Inhaltsverzeichnis
    \tableofcontents
    \newpage

    % erzeige Illustartionsverzeichnis (Abbildungen, tabellen Quellcode)
    \listofillustrations

    %% erzeuge Abkuerzungsverzeichnis
    \listglossaries

    % Aktiviere FancyHeader
    \pagestyle{fancy}	   
    % Arabische Seitenzahlen
    \pagenumbering{arabic}	

    % erstes Kapitel
    \chapter{{\LaTeX}-Vorlage}
Das  hier ist eine kleine {\LaTeX}-Vorlage fuer die DHBW Loerach.
Speziell fuer den fachbereich Informatik.

\section{Danke Dounald}
Dank Dounald Knuth\cite{knuth} der Erfinder von \TeX~ koennen wir hier ganz schoen, tolle und Fancy Ausarbeitungen schreiben ohne einen gedanken an das Formatieren zu verschweden.


\newpage
\section{Aufbau der Vorlage}
Die Vorlage ist von der Standart Vorlage book.cls abgeleitet und ist direkt auf DIN A4 gesetzt.
Auch die Schriftart ist auf 12pt und 1,5 Zeilebabstand gesetzt.

\section{Beispiele}
Nun sollen einzelne Beispiele folgen, wie Bilder, Tabelle, Auflistungen, Programmcode, Zitate 

\subsection{Bilder}
ein Bild wird ganz einfach eingebunden, darum herum wird ein \textit{figure}  gesetzt damit die Abbildung auf auf der Abbildungsauflistung erscheint.\\
Mit \textit{caption} wird eine Abbildungsunterschrift angegeben und mit \textit{label} wird ein label vergeb mit dem man auf eine Abbildung verweisen kann.
\begin{figure}[hb] \centering
  \includegraphics[width=3cm]{./img/tux.png}
  \caption{eine Bild Unterschrift}
  \label{tux_1}
\end{figure}
Als Beispiel eines label kann in \ref{tux_1} gesehen werden das dort ein Pinguin abgebildet wird.


\subsection{Tabelle}
Eine Tabelle wird wie bei einer Abbildung wie folgt dargestellt: \\

\begin{table}[hb] \centering
\begin{tabular}{|r|r|}
\hline
Zahl & Quatsch \\
\hline
1  &  0  \\
\hline
2  &  1  \\
\hline
3  &  2  \\
\hline
4  &  3  \\
\hline

\end{tabular}

\caption{eine QuatschTabelle}
\label{tabquatsch}
\end{table}



\section{Glossar}
Das Glossar listet alle Verwendeten Begriffe auf und verlinkt diese auch im Text. Da ist \gls{LaTeX} echt cool und es macht spass mit einem \gls{computer} zuarbeiten.




    % zweites Kapitel
    \include{chapters/2_dummy}
	


    

    % Ausgabe Literaturverzeichnis
%    \newpage

\markboth{}{}%
\renewcommand{\baselinestretch}{.90}

%Quellen
\begin{thebibliography}{20}
%Reihenfolge: Personen, Internes  B"ucher, Magazine, Internetquellen
%%%%%%%%%   Personen 
\bibitem{myComp}
	Interne Dokumente, Marktanalysen, Strategiepläne und Gespräche mit Mitarbeitern\\
	aus dem eigenen Unternehmen
	<MONAT><jahr>, <Unternehmen Name, ORT>

%%%%%%%%%   Vorlesungsunterlagen
\bibitem{vl}
	Vorlesungsunterlagen <VL-Titel> \\
	Dozent: <Titel Name> \\
	<von> - <bis>,  DHBW-Lörrach

%%%%%%%%%   Buecher 
\bibitem{book}
	<autor> \\
	<Buch Titel> (1962)\\
	<Verlagsname>; Auflage: <wievielete Auflage><ErscheinungsDatum> \\
	ISBN-10: <ISBN Nummer>
	
\bibitem{knuth}
	Donald E. Knuth\\
	The Art of Computer Programming, Volumes 1-4\\
	Addison-Wesley Longman, Amsterdam; Auflage: 3rd edition. (3. März 2011)\\
	ISBN-10: 0321751043




%%%%%%%%%    Magazine
\bibitem{ix_110}
	ix - Magazin fuer Profesionelle Informationstechnik  \\
	1. Ausgabe Januar 2010, Seite 42\\
	Heise Zeitschriften Verlag GmbH \& Co. KG 

%%%%%%%%%    Internetquellen Paper
\bibitem{blogfefe}
	Blog von Felix von Leiter(fefe)\\
	Debian stolpert mal wieder über einen schlechten Zufallszahlengenerator.  \\
	Einsichtnahme: 30.04.2012  \\
	\url{http://blog.fefe.de/?ts=b1628b67}
   
 

\end{thebibliography}

      
    
    \bibliography{./literatur.bib}
    \clearpage
    
    % erzeuge Erklaerung    
    \makestatement

	
\end{document}




